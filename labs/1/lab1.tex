\documentclass{tufte-handout}
\usepackage{amsmath,amsthm}

\usepackage{booktabs}
\usepackage{graphicx}
\usepackage{tikz}

\newtheorem{claim}{Claim}[section]
\title{Marking Trees}
\date{}
\author{}

\begin{document}
\maketitle


Bob maintains a complete binary tree of height $h > 1$ with $N = 2h - 1$ nodes called $1, \ldots , N$. Nodes are in one of two states: marked or unmarked; initially, all nodes are unmarked. Each round Alice sends an integer $i \in \{1, \ldots , N\}$ according to a random process specified below. Bob marks node $i$ in his tree. Bob then uses the following marking rules repeatedly:


\begin{enumerate}
\item A non-leaf node gets marked when both its children are marked, so  \makebox[7mm][r]{\includegraphics[height=5mm]{children}} becomes \makebox[7mm][r]{\includegraphics[height=5mm]{all}}.
\item A non-root node gets marked if its parent and sibling are marked, \makebox[7mm][r]{\includegraphics[height=5mm]{child}} becomes \makebox[7mm][r]{\includegraphics[height=5mm]{all}}.
\end{enumerate}


Bob applies these rules as often as possible, often leading to a cascade of markings.
When no more rules apply, this round ends. For instance, if Alice sends ``5'' in the figure to the right then the entire tree gets marked in this round.
\begin{marginfigure}\includegraphics[height=28mm]{tree}\end{marginfigure}

The whole process ends when every node in Bob’s tree is marked. 

We study 3 random processes:
\begin{enumerate}
\item[$R_1$:] Each round, Alice sends the name of a node independently and uniformly at random from $\{1,\ldots , N\}$.
\item[$R_2$:] Each round, Alice sends the name of a node uniformly at random \emph{from those she has not sent before}. Another way of saying this is that Alice starts by picking a random permutation $\pi$ on $\{1,\ldots, N\}$ and then sends $\pi(i)$ in round $i$.
\item[$R_3$:] Each round, Alice sends the name of a node uniformly at random \emph{from those that are not yet marked}. (Alice can see Bob’s tree.)
\end{enumerate}


To see what you are expected to do this lab, read the lab report template below (including sidenotes).


Hints: For $R_2$ and $R_3$, you may be tempted to pick random numbers uniformly from $\{1,\ldots , N\}$ each round and and throw away those already sent. That takes too much time. Instead, for $R_2$ start with a random permutation using the ``Knuth shuffle'' (a.k.a. ``Fisher and Yates shuffle''). Look it up.

To understand what is going on, maybe it helps you to see which nodes Alice sent. (In particular, what is the last node she sends before the process stops? Statistically, what is the ``bottleneck'' in the process to mark the whole tree.?)




\newpage
\section{Lab Report: Marking Trees}


by Alice Cooper and Bob Marley\sidenote{Complete the report by filling
  in your names and the parts marked $[\cdots]$.
  Remove the sidenotes in your final hand-in.}


\subsection{Results}


For $i \in \{1, 2, 3\}$, the number of rounds $R_i$ spent until the tree is completely marked in process $i$ is given in the following table. The table shows the result of $[\cdots]$ repeated trails.\sidenote{Report your empirical data for
$R_i$. Give each value as the mean plus/minus one standard deviation. Use whatever best practices for reporting data you may have learned; here’s a crash course that suffices
for our purposes: (i) Calculate mean and standard deviation ($m = 2.5074$,
$s = 0.889341021813$) from a number
of repeated experiments. (ii) Round $s$ to one significant digit ($s = 0.9$). (iii) Round $m$ to the decimal place corresponding to the first significant digit in $s$ ($m = 2.5, s = 0.9$). (iv) Report $m \pm s$ $(2.5 \pm 0.9)$. (v) Use scientific notation.\\
If you’ve taken too many statistics classes, feel free to go to town with graphs and confidence intervals and so forth.}

In the last column, report the expected value of $R_1$ for each $N$, using the formula derived from your theoretical analysis in the following section.\sidenote{Use more than one term when evaluating the formula, the result will hopefully be (surprisingly) close to the empirical data.}


\medskip
\begin{tabular}{ccccc}
  \toprule
  $N$ & $R_1$ & $R_2$ & $R_3$ & $\mathbf{E}[R_1]$ \\
  \midrule
  3 & $2.5 \pm 0.9$ & & & \\
  7 & & & & \\
   15 & & & & \\
         31 & & & & \\
         63 & & & & \\
        127 & & & & \\
        255 & & & & \\
        511 & & & & \\
       1023 & $3.2 \pm 0.5 \times 10^{-3}$ & & & \\
  \vdots & & & & \\
  524 287 & $3.2 \pm 0.2 \times 10^{-6}$ & & & \\
  1 048 575 & & & & \\
  \bottomrule
\end{tabular}


\subsection{Analysis}

Our experimental data indicates that $\mathbf{E}[R_1]$ is $[\cdots]$. Optional: $\mathbf{E}[R_2] [\cdots]$, and $\mathbf{E}[R_3] [\cdots]$.\sidenote{For each of the tree processes, try to express the observed behaviour
of $R_i$ using standard terminology from the analysis of algorithms. For instance, use expressions such as ``$\mathbf{E}[R_1]$ is logarithmic in $N$'' or ``$\mathbf{E}[R_2]$ is somewhere between $\Omega(N^{1/2})$ and $\Omega(N^{3/2})$.''}

Theoretically, the behaviour of $R_1$ can be explained as follows: $[\cdots]$.\sidenote{This is the difficult part. You need to write a few lines that explain the random process underlying $R_1$ and derive an expression for $\mathbf{E}[R_1]$.\\
\emph{Hint}: If you’re stuck at $R_1$, do the following experiment as a warm-up. Process 0 is like process 1, except that Bob doesn't use his marking rules:
the only nodes that get marked are those sent by Alice. Implement this (it’s easy) and analyse the behaviour both theoretically and practically.\\
\emph{Hint 2}: Still stuck at $R_1$? Draw sketches of the tree in the beginning, the middle and the end of the process. We’re interested in the ending phase, can you make any assumptions to simplify the problem?\\
\emph{Optional}: Explain the behaviours of $R_2$ and $R_3$ as well. The behaviour of $R_2$ is quite a bit harder; while $R_3$ is just cute.}


\pagebreak

\subsection{Perspective}
This lab establishes minimal skills in simulation of random processes, introduces the Knuth shuffle for those who haven’t seen it, and some basic probability theory about occupancy problems. In process 2, Alice’s messages are not independent, which can lead to temping errors in the analysis. Process 3 is just surprising (and maybe fun to implement), but not much about randomness is to be learned from it.

The exercise is built on an assignment of Michael Mitzenmacher\sidenote{Michael Mitzenmacher, An Experi- mental Assignment on Random Processes, SIGACT News, 27 December 2000. See also section 5.8 in M. Mitzenmacher and E. Upfal, Probability and Computing -- Randomized Algorithms and Probabilistic Analysis. Cambridge University Press, 2005.} as adapted by Thore Husfedlt.


\end{document}
 