\documentclass{tufte-handout}
\usepackage{amsmath,amsthm}

\usepackage{booktabs}
\usepackage{graphicx}
\usepackage{tikz}

\newtheorem{claim}{Claim}[section]
\title{Marking Trees}
\date{}
\author{}

\begin{document}
\maketitle

\section{Lab Report: Marking Trees}


by Alice Cooper and Bob Marley
% \sidenote{Complete the report by filling in your names and the parts marked $[\cdots]$. Remove the sidenotes in your final hand-in.}

\subsection{Results}


For $i \in \{1, 2, 3\}$, the number of rounds $R_i$ spent until the tree is completely marked in process $i$ is given in the following table. The table shows the result of $[\cdots]$ repeated trails.
% \sidenote{Report your empirical data for $R_i$. Give each value as the mean plus/minus one standard deviation. Use whatever best practices for reporting data you may have learned; here’s a crash course that suffices for our purposes: (i) Calculate mean and standard deviation ($m = 2.5074$, $s = 0.889341021813$) from a number of repeated experiments. (ii) Round $s$ to one significant digit ($s = 0.9$). (iii) Round $m$ to the decimal place corresponding to the first significant digit in $s$ ($m = 2.5, s = 0.9$). (iv) Report $m \pm s$ $(2.5 \pm 0.9)$. (v) Use scientific notation.\\ If you’ve taken too many statistics classes, feel free to go to town with graphs and confidence intervals and so forth.}

In the last column, report the expected value of $R_1$ for each $N$, using the formula derived from your theoretical analysis in the following section.
% \sidenote{Use more than one term when evaluating the formula, the result will hopefully be (surprisingly) close to the empirical data.}


\medskip
\begin{tabular}{ccccc}
  \toprule
  $N$     & $R_1$                      & $R_2$                   & $R_3$              & $\mathbf{E}[R_1]$ \\
  \midrule
  3       & $ 4.5 \pm 2.7 $            & $ 2 \pm 0.0 $           & $ 2 \pm 0.0 $                          \\
  7       & $ 8.1 \pm 3.4 $            & $ 4.5 \pm 0.7 $         & $ 4 \pm 0.0 $                          \\
  15      & $ 18.6 \pm 7.5 $           & $ 10.3 \pm 1.7 $        & $ 8 \pm 0.0 $                          \\
  31      & $ 44.6 \pm 17.2 $          & $ 23.1 \pm 3.2 $        & $ 16 \pm 0.0 $                         \\
  63      & $ 110.3 \pm 36.5 $         & $ 50.9 \pm 5.6 $        & $ 32 \pm 0.0 $                         \\
  127     & $ 259.2 \pm 77.5 $         & $ 108.6 \pm 8.5 $       & $ 64 \pm 0.0 $                         \\
  255     & $ 609.5 \pm 168.3 $        & $ 228.6 \pm 12.9 $      & $ 128 \pm 0.0 $                        \\
  511     & $ 1390.3 \pm 312.8 $       & $ 472.9 \pm 19.4 $      & $ 256 \pm 0.0 $                        \\
  1023    & $ 3104.3 \pm 632.4 $       & $ 968.7 \pm 27.6 $      & $ 512 \pm 0.0 $                        \\
  2047    & $ 7053.8 \pm 1370.1 $      & $ 1968.8 \pm 39.1 $     & $ 1024 \pm 0.0 $                       \\
  4095    & $ 15363.3 \pm 2638.0 $     & $ 3981.8 \pm 59.2 $     & $ 2048 \pm 0.0 $                       \\
  8191    & $ 33367.4 \pm 5369.0 $     & $ 8032.8 \pm 85.5 $     & $ 4096 \pm 0.0 $                       \\
  16383   & $ 72923.7 \pm 10347.8 $    & $ 16151.6 \pm 122.2 $   & $ 8192 \pm 0.0 $                       \\
  32767   & $ 154355.2 \pm 18611.3 $   & $ 32434.6 \pm 166.5 $   & $ 16384 \pm 0.0 $                      \\
  65535   & $ 338238.5 \pm 41518.2 $   & $ 65088.0 \pm 237.0 $   & $ 32768 \pm 0.0 $                      \\
  131071  & $ 721874.9 \pm 88828.3 $   & $ 130427.6 \pm 321.8 $  & $ 65536 \pm 0.0 $                      \\
  262143  & $ 1529773.3 \pm 176992.5 $ & $ 261233.0 \pm 481.1 $  & $ 131072 \pm 0.0 $                     \\
  524287  & $ 3214416.7 \pm 321450.5 $ & $ 523013.3 \pm 681.7 $  & $ 262144 \pm 0.0 $                     \\
  1048575 & $ 6822260.0 \pm 651979.4 $ & $ 1046755.5 \pm 961.5 $ & $ 524288 \pm 0.0 $                     \\
  \bottomrule
\end{tabular}


% 1023        & $3.2 \pm 0.5 \times 10^{-3}$ &                         &                    &                   \\
% 524 287     & $3.2 \pm 0.2 \times 10^{-6}$ &                         &                    &                   \\
% 1 048 575   &                              &                         &                    &                   \\

\subsection{Analysis}

Our experimental data indicates that $\mathbf{E}[R_1]$ is $[\cdots]$. Optional: $\mathbf{E}[R_2] [\cdots]$, and $\mathbf{E}[R_3] [\cdots]$.
% \sidenote{For each of the tree processes, try to express the observed behaviour of $R_i$ using standard terminology from the analysis of algorithms. For instance, use expressions such as ``$\mathbf{E}[R_1]$ is logarithmic in $N$'' or ``$\mathbf{E}[R_2]$ is somewhere between $\Omega(N^{1/2})$ and $\Omega(N^{3/2})$.''}

Theoretically, the behaviour of $R_1$ can be explained as follows: $[\cdots]$.
% \sidenote{This is the difficult part. You need to write a few lines that explain the random process underlying $R_1$ and derive an expression for $\mathbf{E}[R_1]$.\\ \emph{Hint}: If you’re stuck at $R_1$, do the following experiment as a warm-up. Process 0 is like process 1, except that Bob doesn't use his marking rules: the only nodes that get marked are those sent by Alice. Implement this (it’s easy) and analyse the behaviour both theoretically and practically.\\ \emph{Hint 2}: Still stuck at $R_1$? Draw sketches of the tree in the beginning, the middle and the end of the process. We’re interested in the ending phase, can you make any assumptions to simplify the problem?\\ \emph{Optional}: Explain the behaviours of $R_2$ and $R_3$ as well. The behaviour of $R_2$ is quite a bit harder; while $R_3$ is just cute.}


\pagebreak

\subsection{Perspective}
This lab establishes minimal skills in simulation of random processes, introduces the Knuth shuffle for those who haven’t seen it, and some basic probability theory about occupancy problems. In process 2, Alice’s messages are not independent, which can lead to temping errors in the analysis. Process 3 is just surprising (and maybe fun to implement), but not much about randomness is to be learned from it.

The exercise is built on an assignment of Michael Mitzenmacher as adapted by Thore Husfedlt.
% \sidenote{Michael Mitzenmacher, An Experi- mental Assignment on Random Processes, SIGACT News, 27 December 2000. See also section 5.8 in M. Mitzenmacher and E. Upfal, Probability and Computing -- Randomized Algorithms and Probabilistic Analysis. Cambridge University Press, 2005.}


\end{document}
